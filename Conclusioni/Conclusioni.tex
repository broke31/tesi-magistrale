\fancyhf{}
\fancyhead[L]{CONCLUSIONI}
\fancyhead[C]{}
\fancyfoot[C]{\thepage}
\pagestyle{fancy}
Nella prima parte della tesi è stata effettuato un mapping study sistematico della letteratura. Questa ricerca ha evidenziato una conoscenza ancora in stato embrionale sull’argomento ed in molti casi una carenza di dati. È emerso che nel tempo sono stati rilasciati pochi dataset sulla problematica trattata e prevalentemente di natura open-source. Inoltre, quasi la totalità degli studi si è concentrata su un unico linguaggio di programmazione (\textbf{Java}) e su un unico sistema di building (\textbf{Maven}). Non sempre sono state descritte in maniera accurata le tecniche utilizzate per la costruzione dei dataset (in alcuni casi sono stati riutilizzati dei dataset preesistenti), inoltre le \emph{root cause} sono state individuate in maniera parziale in quanto in alcuni articoli più recenti ne sono comparse di nuove (soprattutto per sistemi operativi mobile) che non erano ancora state individuate.

Si è quindi deciso di contribuire alla costruzione di un nuovo tool per l’identificazione dei \emph{flaky test} indipendenti dall’ordine. Partendo dal dataset già rilasciato dagli sviluppatori di \textbf{iDFlakies}, sono state analizzati centosessantasei casi di test classificati come non dipendenti dall’ordine. Di questi, trentuno hanno mostrato un comportamento “flaky”. Sono state poi analizzate le root cause di questi metodi ed è emerso che nella quasi totalità dei casi (trenta \emph{flaky test}) la \emph{root cause} era legata al network, mentre soltanto per un \emph{flaky test} la \emph{root cause} era legata al multithreading. Il tool ha permesso di ottenere informazioni più dettagliate su particolari metodi “flaky”, come lo stato della macchina prima e dopo l’esecuzione del test e il timestamp di ogni singola iterazione. Inoltre, il tool ha evidenziato anche i limiti degli attuali sistemi di build che in molti casi risultano poco ottimizzati per affrontare adeguatamente lo studio della problematica in esame.

Infine, i risultati sono stati immagazzinati all’interno di una base di dati e sono
stati rappresentati sotto forma di grafici in modo da evidenziare eventuali pattern. Dai risultati è emerso che in alcuni casi è stato riscontrato effettivamente la presenza di un pattern, aprendo la strada all’ipotesi che alcuni flaky test abbiano un comportamento deterministico. I risultati ottenuti dovranno essere ulteriormente approfonditi.
Gli sviluppi futuri del lavoro saranno:
\begin{itemize}
\item “Instrumentation” del codice per poter monitorare ed avere maggiore controllo sui casi di test che sono in esecuzione;
\item Ampliare il tool in modo da poter individuare flaky test dipendenti dall’ordine;
\item Rendere il tool indipendente dal linguaggio di programmazione in cui sono scritti i progetti in analisi.
\end{itemize}