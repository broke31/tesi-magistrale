Con il propagarsi delle metodologie \emph{Agile}, nelle grandi multinazionali in cui
si sviluppano migliaia di righe di codice ogni giorno, si è reso necessario adottare tecniche di \emph{continuous integration}, le quali permettono agli sviluppatori di
aggiungere nuove funzionalità ad un sistema software già esistente, per poi eseguire
in automatico l’intera suite di test per verificare che le nuove funzionalità non abbiano introdotto \emph{bug} all’interno del software. Se durante questa fase non viene riscontrato il fallimento di nessun caso di test, il software potrà essere reso di nuovo
disponibile per ulteriori modifiche, in caso contrario invece bisognerà individuare il difetto introdotto e correggerlo. Non sempre però il fallimento di un caso di test è causato dell’introduzione di nuovi \emph{bug} all’interno del software, infatti questo può
derivare dalla presenza di un metodo \emph{flaky} presente all’interno della suite di test.

Un test si definisce \emph{flaky} se il suo comportamento non può essere stabilito in \textbf{maniera deterministica}. Appare quindi evidente che la presenza di un metodo \emph{flaky}
all’interno della test suite possa incidere non solo sulla qualità, ma anche sui tempi di rilascio e sui costi del software. Inoltre, analisi condotte in diversi studi
focalizzati sui \emph{flaky test}, hanno mostrato che una delle criticità più frequenti quando si manifesta un \emph{flaky} è capire che cosa l’ha scatenato.

L’obiettivo che ci si pone all’interno di questa tesi è lo sviluppo di un nuovo tool che possa aiutare gli sviluppatori ad identificare la presenza di \emph{flaky} all’interno
dei loro progetti e a individuare la sua \emph{root cause}. Infatti, definire la \emph{root cause}
potrebbe rappresentare il primo passo per riuscire a mitigare la problematica dei
\emph{flaky}. Il tool sviluppato è in grado di individuare \emph{flaky} test, la cui \emph{root cause} non è
la dipendenza dall’ordine di esecuzione, all’interno di progetti scritti in \textbf{Java} e che fanno uso di \textbf{Maven} come building system. Il tool proposto esegue un certo numero
di volte uno specifico caso di test, salvando non solo il risultato ottenuto dall’esecuzione (pass o fail), ma anche delle informazioni aggiuntive legate allo stato della macchina su cui si sta eseguendo il test e il tempo di esecuzione del singolo caso di test.

Queste due informazioni sono state aggiunte poiché non venivano riportate in nessuno dei dataset precedentemente rilasciati, ma possono essere utili per fare delle analisi a grana fine per l'individuazione delle \emph{root cause} e di eventuali pattern.
Infine, oltre alla generazione di un file csv con all’interno le informazioni descritte in precedenza, viene anche generato un file aggiuntivo in cui si va a tenere traccia dell’esito della build di ogni progetto analizzato.

Per valutare questo tool è stato considerato il dataset rilasciato dagli
sviluppatori di \textbf{iDFlakies}, un framework in grado di individuare un \emph{flaky test} e fare
una classificazione parziale delle \emph{root cause} per i \emph{flaky} individuati (indipendente
dall’ordine di esecuzione e dipendente dall’ordine di esecuzione). Da questo dataset sono state selezionate centosessantasei righe, ognuna rappresentante un caso di test
\emph{flaky} la cui \emph{root cause} era diversa dalla dipendenza dall’ordine di esecuzione del
test.

Infine, per ognuno dei metodi \emph{flaky} rilevati da questo tool (trentuno test) è stata fatta un'analisi statica del codice sorgente per identificare la \emph{root cause}. Da questa analisi è emerso che per trenta \emph{flaky test} la \emph{root cause}era riconducibile a problemi di \textbf{network}, mentre in un solo caso è stata identificata come root cause il \textbf{multithreading}.

Una volta terminata l’analisi, si è passati ad immagazzinare i dati ottenuti
all’interno di una base di dati ed a creare dei grafici sui risultati delle singole
esecuzioni di un caso di test, in modo da evidenziare possibili pattern. Sono quindi
emersi in alcuni casi dei comportamenti “deterministici” del \emph{flaky test } che dovranno essere ulteriormente approfonditi.

La tesi è organizzata nel seguente modo.

Nel primo capitolo viene presentata una panoramica generale del testing, soffermandosi in particolare sul testing di regressione, sui sistemi di build e sulla \emph{continuous integration}.

Nel secondo capitolo viene presentata un mapping study sistematico della letteratura dei \emph{flaky test}, che ha permesso di comprendere a fondo lo stato dell’arte e di identificare gli approcci che sono stati utilizzati in questi anni per far fronte a
questa problematica.

Nel terzo capitolo è presentata la costruzione di un tool che possa essere di supporto agli sviluppatori per aiutarli ad individuare metodi \emph{“flaky”} all’interno dei loro progetti. Tale tool va ad estendere un dataset già presente in letteratura
aggiungendo nuove informazioni che potrebbero essere d’interesse per eventuali ricerche future.

Le conclusioni riassumono i risultati ottenuti e contengono gli sviluppi futuri che potranno essere messi in pratica per ampliare la conoscenza sui \emph{flaky test}.
